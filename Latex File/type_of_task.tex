\section{Introduction}
As we all know, in real-time computing systems, there are numerous of complex control applications and tasks that need to be finished off within a certain time constraints. This is what we called as deadlines. If complying the deadline is critical or strict for the operation of system which later may result in a catastrophic ramification if the deadline is not met, then the deadline is regarded to be hard. On the other hand, if meeting the deadline is desirable but missing it would not cause any critical damage, then the deadline is said to be soft. One easy example for soft real time would be multimedia applications such as audio or video streaming as for these applications, missing a deadline only leads to the degradation of the system performance and has no catastrophic consequences. In addition to their criticalness, the tasks are divided into three which are periodic, aperiodic and sporadic. Basically, tasks that are activated or initiated regularly are called periodic tasks meanwhile, tasks which have irregular arrival times are called aperiodic and aperiodic tasks with hard deadlines are called sporadic tasks. 

We now contemplate that a real-time system consists of a firm set of periodic tasks and sporadic tasks that may occur dynamically. Each periodic task will generate service requests periodically or simply at a predictable time. The attributes of every periodic task, $T_{i}$ of a static timing includes its worst execution time $q$, its period $P_{i}$ and its critical delay $R_{i}$ which presume to be equal or less than the period. To the contrary, sporadic tasks arise at the node at unpredictable times. Every sporadic task will be ready to be processed once the request is being granted or accepted on the node \cite{b5}. 

A set of concurrent real-time tasks must be executed by hard real-time system to the extent that all time-critical tasks satisfy the deadlines that have been set for them. Each task requires some data on the computational and other resources like the input or output of a device to carry on the process. The scheduling problem is burdened by the allocation of these resources to satisfy all timing requirements \cite{b3}. Now that we already been informed about the basic knowledge of tasks, we will now go through the common scheduling problems and also the type or fractions of scheduling used in real-time scheduling.
