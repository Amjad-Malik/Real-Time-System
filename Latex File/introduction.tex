\section{Introduction}
Real-time systems are computing systems that must react within precise time constraints to events in the environment \cite{b2}. In respect of this matter, the correct behaviour of the systems does not only be dependent upon the logical value of the computation, but also influenced by the time which the outcomes are produced. A response that happens too late or simply within the specific designed time or period may harm or produce a pointless outcome. Therefore, there is no doubt that real-time computing plays a big and important part in the real world. As we can see clearly throughout the years, the number of complex systems that lean on in either partially or wholly on computer control is expanding widely. Some examples of the applications that uses real-time computing are telecommunication system, nuclear and chemical plant, flight control system, industrial automation, modern medical system, railway system and many more \cite{b2}.

This topic of real time systems is broad and vast but, in this paper, we will only be focusing on certain algorithms included in the dynamic scheduling, but other important elements will also be explained briefly. The main objective behind this research studies is to learn some efficient algorithms for the joint scheduling of aperiodic requests and hard periodic tasks under the Earliest Deadline First (EDF) policy. Besides, we can also learn and understand the importance of all those stated algorithms which establish a helpful foundation and framework in accommodating a Hard Real Time (HRT) system designer in specifying the most suitable method in accordance with his or her needs, by balancing the efficiency of the algorithms with implementation overhead [8]. This research studies touches on three optimal algorithms that possess different implementation overheads and performances based upon the desired usage. The algorithms that will be included are the Total Bandwidth Server (TBS), Earliest Deadline Latest (EDL) server and not to forget, Constant Bandwidth Server (CBS). 

The algorithms stated before did intensify the preceding servers introduced before them greatly in terms of the performance and cost ratio. Not only that, they also can be easily implemented with very little overheads. But before that, it is good to have a grasp on the basic idea on the real time systems. For a better understanding on this topic, the flow of this paper will start with the explanation of the type of tasks that used in the system which have been categorized by the engineers and designers throughout the development, the problems and type of real-time scheduling, the insights for the three servers stated before, the simulation and comparison between servers and finally the implementation of the CBS in a software called UPPAAL.  
